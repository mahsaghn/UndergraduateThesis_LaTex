

\section{Literature Review}
Overview previous works and papers.
\section{Introduction}

\section{Experiments}
ParsFEVER Code + Another paper codes and mutations on that code. 
\section{Results}
\section{Conclusion}

\cite{stance_survey}
An additional challenge is that existing
datasets also have certain potentially undesirable biases.
Datasets constructed from naturally occurring data in fact-
checking portals, such as Augenstein et al. [2019] exhibit a
long-tail distribution when it comes to which entities are men-
tioned in the claims. Moreover, due to the nature of which
claims are deemed to be worthy for fact-checking, some en-
tities can be mentioned in claims with predominantly one
veracity class. This, in turn, could yield biased models and
biased decisions when such models are put in practical use.
Another option is to use artificially constructed datasets such
as FEVER, which are also biased as the way the claims and
the evidence statements are expressed is not representative
of naturally occurring claims, and it is likely that the types
of claims are not reflective of check-worthy claims either
[Wright and Augenstein, 2020]. As biases in datasets are un-
avoidable, it might instead be worth noting what the intended
application for them is [Waseem et al., 2020].



% If you are copying and pasting material from one of your papers, then remember to:
% \begin{itemize}
%     \item Remove the abstract and instead add a little overview of the chapter and how it ties in to the rest of the thesis. You should also mention the original paper's source like: ``This chapter includes materials originally published in $\backslash$citet\{myownppr\}''
%     \item Make sure the formatting still works -- this is single column now!
%     \item Consider rephrasing conference-paper-style language:
%     \begin{itemize}
%         \item Find every place you mention some variation of ``in this paper'' and say ``in this chapter'' instead.
%         \item Remove or rephrase the parts where you talk about ``our main contributions''.
%         \item Rephrase the language describing code and data releases.
%     \end{itemize}
%     \item Replace the conclusion section with a summary section. Again, you should tie this chapter back to the main themes of the thesis.
% \end{itemize}



% \begin{wrapfigure}{r}{0.3\textwidth}
%     \centering
%     \includegraphics[width=.3\textwidth]{example-image-b}
%     \caption[Short title for wrapfigure]{You can also make inline figures.}
%     \label{chap2:fig:ex2}
% \end{wrapfigure}

% Here are a few notes about the layout and usage of this latex template:

% \paragraph{Captions} In order for the captions from figures and tables to show up cleanly in the lists of figures and tables, you should use the caption command with the bracket argument to create a short title for the list of figures/tables, like:

% \texttt{$\backslash$caption [shortened title]\{full caption\}}

% \noindent as in Figure~\ref{chap2:fig:ex} and Table~\ref{chap2:tab:ex}.

% \paragraph{Citations} This latex was given to me with an old NAACL style file that uses the standard \texttt{$\backslash$citep}  and \texttt{$\backslash$citet}  commands as in Table~\ref{chap2:tab:ex}.  I don't think it supports \texttt{$\backslash$cite} or \texttt{$\backslash$newcite}. Feel free to add commands as needed for you.

% \begin{figure}[tb]
%     \centering
%     \includegraphics[width=.5\textwidth]{example-image-a}
%     \caption[Short title  - only appears in list of figures]{Full caption, make it as long as you want.}
%     \label{chap2:fig:ex}
% \end{figure}


% \begin{table}[tb]
%     \centering
%     \begin{tabular}{lll}
%     \toprule
%          Latex command & Citation\\
%     \midrule
%         $\backslash$citep example & \citep{grice1975logic} \\
%         $\backslash$citet example & \citet{grice1975logic} \\
%     \bottomrule
%     \end{tabular}
%     \caption[Citation examples]{The style files that come included in this latex template use $\backslash$citep and $\backslash$citet.} 
%     \label{chap2:tab:ex}
% \end{table}