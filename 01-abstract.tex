%% Replace all of the orange portions with your personal info

\thispagestyle{empty}
\begin{centering}
\vspace{1in}
Iran University of Science and Technology \\
\vspace*{1.\baselineskip}
{\bf Abstract}\\
\vspace*{1\baselineskip}

{\thesisTitle}\\ %self-explanatory
\vspace*{1.\baselineskip}
{\authorName} \\ %self-explanatory
\vspace*{1.\baselineskip}


\ifdefined\secondAdvisor
    Co-chairs
    \else
    Chair
\fi
of the Supervisory Committee:\\ %change to co-chair if co-advised 
\advisorTitle~\advisor\\ \vspace{-.5em} \advisorDepartment \\
\ifdefined\secondAdvisor
    \secondAdvisorTitle~\secondAdvisor\\\vspace{-.5em}\secondAdvisorDepartment \\
\fi
\end{centering}
\vspace*{\baselineskip}
These days increase in unauthenticated sources on the Internet, has led to spreading fake news vastly. 
Failure to detect misinformation promptly can have significantly irreparable damages on the walk of life.
Recent researches have improved stance classification as a primitive step to detect fake news in English and they have less focused on detecting fake news.
In this work, we have developed deep learning models to improve the accuracy of Persian stance classification. 
Then, we implemented a model to detect Persian fake news by using our best stance classier and other manually extracted features. 
Consequently, we achieve an accuracy of 84\% for stance classification and 99\% for fake news detection on the employed dataset.
