%% Replace all of the orange portions with your personal info

\thispagestyle{empty}
\begin{centering}
\vspace{1in}
Iran University of Science and Technology \\
\vspace*{1.\baselineskip}
{\bf Abstract}\\
\vspace*{1\baselineskip}

{\thesisTitle}\\ %self-explanatory
\vspace*{1.\baselineskip}
{\authorName} \\ %self-explanatory
\vspace*{1.\baselineskip}


\ifdefined\secondAdvisor
    Co-chairs
    \else
    Chair
\fi
of the Supervisory Committee:\\ %change to co-chair if co-advised 
\advisorTitle~\advisor\\ \vspace{-.5em} \advisorDepartment \\
\ifdefined\secondAdvisor
    \secondAdvisorTitle~\secondAdvisor\\\vspace{-.5em}\secondAdvisorDepartment \\
\fi
\end{centering}
\vspace*{\baselineskip}
These days increase in unauthenticated internet sources has led to spreading fake news vastly. Failure to detect misinformation promptly can have significantly irreparable damages on the walk of life. Recent researches have improved stance classification as a primitive step to detect fake news in English. Moreover, they have less focus on recognizing fake news. In this work, we have developed a deep learning model based on ParsBERT to improve the accuracy of the Persian stance classification. We over-sampled the available imbalanced dataset in Persian to compensate lack of data in such a low-resource language. Then, we implemented a model to detect fake news in Persian by using our best stance classifier and other manually extracted features. Consequently, we achieved an accuracy of 85\% for stance classification and 99\% for fake news detection on the employed dataset.

\bigbreak
\vspace*{3cm}
\bigbreak
Keywords: Machine Learning, Deep Learning, Natural Language Processing, BERT, Oversampling, Stance Classification, Fake News Detection